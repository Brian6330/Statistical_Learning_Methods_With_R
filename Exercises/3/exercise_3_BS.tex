% Options for packages loaded elsewhere
\PassOptionsToPackage{unicode}{hyperref}
\PassOptionsToPackage{hyphens}{url}
%
\documentclass[
]{article}
\usepackage{amsmath,amssymb}
\usepackage{lmodern}
\usepackage{iftex}
\ifPDFTeX
  \usepackage[T1]{fontenc}
  \usepackage[utf8]{inputenc}
  \usepackage{textcomp} % provide euro and other symbols
\else % if luatex or xetex
  \usepackage{unicode-math}
  \defaultfontfeatures{Scale=MatchLowercase}
  \defaultfontfeatures[\rmfamily]{Ligatures=TeX,Scale=1}
\fi
% Use upquote if available, for straight quotes in verbatim environments
\IfFileExists{upquote.sty}{\usepackage{upquote}}{}
\IfFileExists{microtype.sty}{% use microtype if available
  \usepackage[]{microtype}
  \UseMicrotypeSet[protrusion]{basicmath} % disable protrusion for tt fonts
}{}
\makeatletter
\@ifundefined{KOMAClassName}{% if non-KOMA class
  \IfFileExists{parskip.sty}{%
    \usepackage{parskip}
  }{% else
    \setlength{\parindent}{0pt}
    \setlength{\parskip}{6pt plus 2pt minus 1pt}}
}{% if KOMA class
  \KOMAoptions{parskip=half}}
\makeatother
\usepackage{xcolor}
\IfFileExists{xurl.sty}{\usepackage{xurl}}{} % add URL line breaks if available
\IfFileExists{bookmark.sty}{\usepackage{bookmark}}{\usepackage{hyperref}}
\hypersetup{
  pdftitle={Exercise \#3},
  pdfauthor={Brian Schweigler; 16-102-071},
  hidelinks,
  pdfcreator={LaTeX via pandoc}}
\urlstyle{same} % disable monospaced font for URLs
\usepackage[margin=1in]{geometry}
\usepackage{color}
\usepackage{fancyvrb}
\newcommand{\VerbBar}{|}
\newcommand{\VERB}{\Verb[commandchars=\\\{\}]}
\DefineVerbatimEnvironment{Highlighting}{Verbatim}{commandchars=\\\{\}}
% Add ',fontsize=\small' for more characters per line
\usepackage{framed}
\definecolor{shadecolor}{RGB}{248,248,248}
\newenvironment{Shaded}{\begin{snugshade}}{\end{snugshade}}
\newcommand{\AlertTok}[1]{\textcolor[rgb]{0.94,0.16,0.16}{#1}}
\newcommand{\AnnotationTok}[1]{\textcolor[rgb]{0.56,0.35,0.01}{\textbf{\textit{#1}}}}
\newcommand{\AttributeTok}[1]{\textcolor[rgb]{0.77,0.63,0.00}{#1}}
\newcommand{\BaseNTok}[1]{\textcolor[rgb]{0.00,0.00,0.81}{#1}}
\newcommand{\BuiltInTok}[1]{#1}
\newcommand{\CharTok}[1]{\textcolor[rgb]{0.31,0.60,0.02}{#1}}
\newcommand{\CommentTok}[1]{\textcolor[rgb]{0.56,0.35,0.01}{\textit{#1}}}
\newcommand{\CommentVarTok}[1]{\textcolor[rgb]{0.56,0.35,0.01}{\textbf{\textit{#1}}}}
\newcommand{\ConstantTok}[1]{\textcolor[rgb]{0.00,0.00,0.00}{#1}}
\newcommand{\ControlFlowTok}[1]{\textcolor[rgb]{0.13,0.29,0.53}{\textbf{#1}}}
\newcommand{\DataTypeTok}[1]{\textcolor[rgb]{0.13,0.29,0.53}{#1}}
\newcommand{\DecValTok}[1]{\textcolor[rgb]{0.00,0.00,0.81}{#1}}
\newcommand{\DocumentationTok}[1]{\textcolor[rgb]{0.56,0.35,0.01}{\textbf{\textit{#1}}}}
\newcommand{\ErrorTok}[1]{\textcolor[rgb]{0.64,0.00,0.00}{\textbf{#1}}}
\newcommand{\ExtensionTok}[1]{#1}
\newcommand{\FloatTok}[1]{\textcolor[rgb]{0.00,0.00,0.81}{#1}}
\newcommand{\FunctionTok}[1]{\textcolor[rgb]{0.00,0.00,0.00}{#1}}
\newcommand{\ImportTok}[1]{#1}
\newcommand{\InformationTok}[1]{\textcolor[rgb]{0.56,0.35,0.01}{\textbf{\textit{#1}}}}
\newcommand{\KeywordTok}[1]{\textcolor[rgb]{0.13,0.29,0.53}{\textbf{#1}}}
\newcommand{\NormalTok}[1]{#1}
\newcommand{\OperatorTok}[1]{\textcolor[rgb]{0.81,0.36,0.00}{\textbf{#1}}}
\newcommand{\OtherTok}[1]{\textcolor[rgb]{0.56,0.35,0.01}{#1}}
\newcommand{\PreprocessorTok}[1]{\textcolor[rgb]{0.56,0.35,0.01}{\textit{#1}}}
\newcommand{\RegionMarkerTok}[1]{#1}
\newcommand{\SpecialCharTok}[1]{\textcolor[rgb]{0.00,0.00,0.00}{#1}}
\newcommand{\SpecialStringTok}[1]{\textcolor[rgb]{0.31,0.60,0.02}{#1}}
\newcommand{\StringTok}[1]{\textcolor[rgb]{0.31,0.60,0.02}{#1}}
\newcommand{\VariableTok}[1]{\textcolor[rgb]{0.00,0.00,0.00}{#1}}
\newcommand{\VerbatimStringTok}[1]{\textcolor[rgb]{0.31,0.60,0.02}{#1}}
\newcommand{\WarningTok}[1]{\textcolor[rgb]{0.56,0.35,0.01}{\textbf{\textit{#1}}}}
\usepackage{graphicx}
\makeatletter
\def\maxwidth{\ifdim\Gin@nat@width>\linewidth\linewidth\else\Gin@nat@width\fi}
\def\maxheight{\ifdim\Gin@nat@height>\textheight\textheight\else\Gin@nat@height\fi}
\makeatother
% Scale images if necessary, so that they will not overflow the page
% margins by default, and it is still possible to overwrite the defaults
% using explicit options in \includegraphics[width, height, ...]{}
\setkeys{Gin}{width=\maxwidth,height=\maxheight,keepaspectratio}
% Set default figure placement to htbp
\makeatletter
\def\fps@figure{htbp}
\makeatother
\setlength{\emergencystretch}{3em} % prevent overfull lines
\providecommand{\tightlist}{%
  \setlength{\itemsep}{0pt}\setlength{\parskip}{0pt}}
\setcounter{secnumdepth}{-\maxdimen} % remove section numbering
\ifLuaTeX
  \usepackage{selnolig}  % disable illegal ligatures
\fi

\title{Exercise \#3}
\usepackage{etoolbox}
\makeatletter
\providecommand{\subtitle}[1]{% add subtitle to \maketitle
  \apptocmd{\@title}{\par {\large #1 \par}}{}{}
}
\makeatother
\subtitle{t-test and R programming}
\author{Brian Schweigler; 16-102-071}
\date{23/03/2022}

\begin{document}
\maketitle

\hypertarget{preliminaries}{%
\subsection{Preliminaries}\label{preliminaries}}

Loading the mean dataset:

\begin{Shaded}
\begin{Highlighting}[]
\FunctionTok{setwd}\NormalTok{(}\FunctionTok{dirname}\NormalTok{(rstudioapi}\SpecialCharTok{::}\FunctionTok{getActiveDocumentContext}\NormalTok{()}\SpecialCharTok{$}\NormalTok{path))}

\NormalTok{mean20\_df }\OtherTok{=} \FunctionTok{read.csv}\NormalTok{(}\StringTok{"Mean20.txt"}\NormalTok{, }\AttributeTok{header =} \ConstantTok{TRUE}\NormalTok{, }\AttributeTok{comment.char =} \StringTok{"\#"}\NormalTok{)}
\end{Highlighting}
\end{Shaded}

Handle the outliers:

\begin{itemize}
\item
  Remove NAs
\item
  Remove illogical values (\textless= 0)
\end{itemize}

\begin{Shaded}
\begin{Highlighting}[]
\NormalTok{temp\_mean20\_df }\OtherTok{\textless{}{-}}\NormalTok{ mean20\_df}
\NormalTok{temp\_mean20\_df[temp\_mean20\_df }\SpecialCharTok{\textless{}=} \DecValTok{0}\NormalTok{] }\OtherTok{\textless{}{-}} \ConstantTok{NA}
\NormalTok{cleaned\_mean20\_df }\OtherTok{\textless{}{-}} \FunctionTok{na.omit}\NormalTok{(temp\_mean20\_df)}
\FunctionTok{summary}\NormalTok{(cleaned\_mean20\_df)}
\end{Highlighting}
\end{Shaded}

\begin{verbatim}
##       time      
##  Min.   :6.850  
##  1st Qu.:6.968  
##  Median :7.010  
##  Mean   :7.008  
##  3rd Qu.:7.072  
##  Max.   :7.120
\end{verbatim}

\begin{Shaded}
\begin{Highlighting}[]
\FunctionTok{sd}\NormalTok{(cleaned\_mean20\_df}\SpecialCharTok{$}\NormalTok{time)}
\end{Highlighting}
\end{Shaded}

\begin{verbatim}
## [1] 0.07515598
\end{verbatim}

\newpage

\hypertarget{suppose-that-mean-delay-between-two-calls-is-7.05-minutes.-can-you-test-this-hypothesis-using-the-available-data-what-is-your-conclusion-do-you-see-a-difference-when-considering-the-original-values-and-the-preprocessed-values}{%
\subsection{2. Suppose that mean delay between two calls is 7.05
minutes. Can you test this hypothesis using the available data? What is
your conclusion? Do you see a difference when considering the original
values and the preprocessed
values?}\label{suppose-that-mean-delay-between-two-calls-is-7.05-minutes.-can-you-test-this-hypothesis-using-the-available-data-what-is-your-conclusion-do-you-see-a-difference-when-considering-the-original-values-and-the-preprocessed-values}}

\begin{Shaded}
\begin{Highlighting}[]
\NormalTok{assumed\_mean }\OtherTok{\textless{}{-}} \FloatTok{7.05}
\NormalTok{estimated\_mean }\OtherTok{\textless{}{-}} \FunctionTok{mean}\NormalTok{(cleaned\_mean20\_df}\SpecialCharTok{$}\NormalTok{time)}
\FunctionTok{t.test}\NormalTok{(}\AttributeTok{x =}\NormalTok{ cleaned\_mean20\_df, }\AttributeTok{mu =}\NormalTok{ assumed\_mean, }\AttributeTok{conf.level =} \FloatTok{0.95}\NormalTok{)}
\end{Highlighting}
\end{Shaded}

\begin{verbatim}
## 
##  One Sample t-test
## 
## data:  cleaned_mean20_df
## t = -2.4992, df = 19, p-value = 0.02178
## alternative hypothesis: true mean is not equal to 7.05
## 95 percent confidence interval:
##  6.972826 7.043174
## sample estimates:
## mean of x 
##     7.008
\end{verbatim}

Assuming we want the 95\% confidance interval, we reject the hypothesis
of the mean delay being 7.05 minutes.

Comparing to the unprocessed values:

\begin{Shaded}
\begin{Highlighting}[]
\FunctionTok{t.test}\NormalTok{(}\AttributeTok{x =}\NormalTok{ mean20\_df, }\AttributeTok{mu =}\NormalTok{ assumed\_mean, }\AttributeTok{conf.level =} \FloatTok{0.95}\NormalTok{)}
\end{Highlighting}
\end{Shaded}

\begin{verbatim}
## 
##  One Sample t-test
## 
## data:  mean20_df
## t = -1.0626, df = 20, p-value = 0.3006
## alternative hypothesis: true mean is not equal to 7.05
## 95 percent confidence interval:
##  4.947647 7.733306
## sample estimates:
## mean of x 
##  6.340476
\end{verbatim}

While for the unprocessed data the true mean is not 7.05, it is clear
that there is a \textgreater95\% probability that our hypothesis holds.
This is because the unprocessed data includes a negative value (I'd
assume NA's are ignored), which increases the total range of values
immensly.

\newpage

\hypertarget{for-mary-the-delay-cannot-be-smaller-than-7.05-minutes.-thus-the-only-credible-alternative-hypothesis-must-take-account-of-this-well-known-fact.-how-can-you-test-marys-hypothesis}{%
\subsection{3. For Mary, the delay cannot be smaller than 7.05 minutes.
Thus the only credible alternative hypothesis must take account of this
(well-known) fact. How can you test Mary's
hypothesis?}\label{for-mary-the-delay-cannot-be-smaller-than-7.05-minutes.-thus-the-only-credible-alternative-hypothesis-must-take-account-of-this-well-known-fact.-how-can-you-test-marys-hypothesis}}

We can use a one-sided t-test for this:

\begin{Shaded}
\begin{Highlighting}[]
\FunctionTok{t.test}\NormalTok{(}\AttributeTok{x =}\NormalTok{ cleaned\_mean20\_df, }\AttributeTok{alternative =} \StringTok{"g"}\NormalTok{, }\AttributeTok{mu =}\NormalTok{ assumed\_mean, }\AttributeTok{conf.level =} \FloatTok{0.95}\NormalTok{)}
\end{Highlighting}
\end{Shaded}

\begin{verbatim}
## 
##  One Sample t-test
## 
## data:  cleaned_mean20_df
## t = -2.4992, df = 19, p-value = 0.9891
## alternative hypothesis: true mean is greater than 7.05
## 95 percent confidence interval:
##  6.978941      Inf
## sample estimates:
## mean of x 
##     7.008
\end{verbatim}

So for the processed data, this hypothesis holds in at least 95\% of the
cases.

\begin{Shaded}
\begin{Highlighting}[]
\FunctionTok{t.test}\NormalTok{(}\AttributeTok{x =}\NormalTok{ mean20\_df, }\AttributeTok{alternative =} \StringTok{"g"}\NormalTok{, }\AttributeTok{mu =}\NormalTok{ assumed\_mean, }\AttributeTok{conf.level =} \FloatTok{0.95}\NormalTok{)}
\end{Highlighting}
\end{Shaded}

\begin{verbatim}
## 
##  One Sample t-test
## 
## data:  mean20_df
## t = -1.0626, df = 20, p-value = 0.8497
## alternative hypothesis: true mean is greater than 7.05
## 95 percent confidence interval:
##  5.188856      Inf
## sample estimates:
## mean of x 
##  6.340476
\end{verbatim}

For the unprocessed variant this also holds true as the range of the
95\% confidence interval is larger.

\newpage

\hypertarget{define-a-function-secondmaxx-where-x-is-a-vector-returning-the-second-largest-value-contained-in-x.-if-x-is-not-a-vector-return-an-error-message.-test-your-implementation-in-different-cases-using-the-mean20-dataset.}{%
\subsection{4. Define a function secondMax(x), where x is a vector,
returning the second largest value contained in x. If x is not a vector,
return an error message. Test your implementation in different cases
using the Mean20
dataset.}\label{define-a-function-secondmaxx-where-x-is-a-vector-returning-the-second-largest-value-contained-in-x.-if-x-is-not-a-vector-return-an-error-message.-test-your-implementation-in-different-cases-using-the-mean20-dataset.}}

First we define the function and check if x is a vector

\begin{Shaded}
\begin{Highlighting}[]
\NormalTok{secondMax }\OtherTok{\textless{}{-}} \ControlFlowTok{function}\NormalTok{(x)}
\NormalTok{\{}
    \ControlFlowTok{if}\NormalTok{ (}\SpecialCharTok{!}\FunctionTok{is.vector}\NormalTok{(x))  \{}
        \FunctionTok{return}\NormalTok{(}\StringTok{"Input x is not a numeric vector!"}\NormalTok{)}
\NormalTok{    \}}
    \CommentTok{\# need at least 2 values}
    \ControlFlowTok{if}\NormalTok{ (}\FunctionTok{length}\NormalTok{(x) }\SpecialCharTok{\textless{}=} \DecValTok{1}\NormalTok{)     \{}
        \FunctionTok{return}\NormalTok{(}\StringTok{"x must contain at least 2 values!"}\NormalTok{)}
\NormalTok{    \}}
    \ControlFlowTok{if}\NormalTok{ (}\SpecialCharTok{!}\FunctionTok{is.numeric}\NormalTok{(x)) \{}
        \FunctionTok{return}\NormalTok{(}\StringTok{"Vector components must be numeric!"}\NormalTok{)}
\NormalTok{    \}}
    
    \CommentTok{\# Remove largest value, such that the second largest value is now the largest}
\NormalTok{    x\_without\_max }\OtherTok{\textless{}{-}}\NormalTok{ x[}\SpecialCharTok{{-}}\FunctionTok{which}\NormalTok{(x }\SpecialCharTok{==} \FunctionTok{max}\NormalTok{(x))]}
    
    \FunctionTok{return}\NormalTok{(}\FunctionTok{max}\NormalTok{(x\_without\_max))}
\NormalTok{\}}
\end{Highlighting}
\end{Shaded}

Testing the implementation:

\begin{Shaded}
\begin{Highlighting}[]
\FunctionTok{secondMax}\NormalTok{(cleaned\_mean20\_df}\SpecialCharTok{$}\NormalTok{time)}
\end{Highlighting}
\end{Shaded}

\begin{verbatim}
## [1] 7.1
\end{verbatim}

\begin{Shaded}
\begin{Highlighting}[]
\FunctionTok{secondMax}\NormalTok{(}\FunctionTok{c}\NormalTok{(}\StringTok{"1"}\NormalTok{,}\StringTok{"2"}\NormalTok{,}\StringTok{"4"}\NormalTok{)) }\CommentTok{\# should fail}
\end{Highlighting}
\end{Shaded}

\begin{verbatim}
## [1] "Vector components must be numeric!"
\end{verbatim}

\begin{Shaded}
\begin{Highlighting}[]
\FunctionTok{secondMax}\NormalTok{(}\FunctionTok{c}\NormalTok{(}\StringTok{"yes"}\NormalTok{,}\StringTok{"this"}\NormalTok{,}\StringTok{"is"}\NormalTok{,}\StringTok{"patrick"}\NormalTok{)) }\CommentTok{\# should fail}
\end{Highlighting}
\end{Shaded}

\begin{verbatim}
## [1] "Vector components must be numeric!"
\end{verbatim}

\begin{Shaded}
\begin{Highlighting}[]
\FunctionTok{secondMax}\NormalTok{(cleaned\_mean20\_df) }\CommentTok{\# should fail}
\end{Highlighting}
\end{Shaded}

\begin{verbatim}
## [1] "Input x is not a numeric vector!"
\end{verbatim}

\begin{Shaded}
\begin{Highlighting}[]
\FunctionTok{secondMax}\NormalTok{(}\ConstantTok{NA}\NormalTok{) }\CommentTok{\# should fail}
\end{Highlighting}
\end{Shaded}

\begin{verbatim}
## [1] "x must contain at least 2 values!"
\end{verbatim}

\begin{Shaded}
\begin{Highlighting}[]
\FunctionTok{secondMax}\NormalTok{(}\FunctionTok{c}\NormalTok{(}\ConstantTok{NA}\NormalTok{, }\ConstantTok{NA}\NormalTok{, }\ConstantTok{NA}\NormalTok{)) }\CommentTok{\# should fail}
\end{Highlighting}
\end{Shaded}

\begin{verbatim}
## [1] "Vector components must be numeric!"
\end{verbatim}

\newpage

\hypertarget{define-a-function-mysummaryx-where-x-is-a-vector-composed-by-the-mean-the-median-the-standard-deviation-the-minimum-and-the-maximum-values-in-this-order.-test-your-implementation-in-different-cases-using-the-mean20-dataset}{%
\subsection{5. Define a function mySummary(x), where x is a vector
composed by the mean, the median, the standard deviation, the minimum
and the maximum values (in this order). Test your implementation in
different cases using the Mean20
dataset}\label{define-a-function-mysummaryx-where-x-is-a-vector-composed-by-the-mean-the-median-the-standard-deviation-the-minimum-and-the-maximum-values-in-this-order.-test-your-implementation-in-different-cases-using-the-mean20-dataset}}

TODO

\end{document}
